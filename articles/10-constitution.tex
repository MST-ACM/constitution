
\section{X. - Constitution}
Note:  This section should specify how the constitution is approved and amended,
and the specific process for introducing, voting on, and approving amendments
and by-laws.
\begin{enumerate}
  \item Adoption
    \begin{enumerate}
      \item	(Specify vote and/or process carried out by the organization members
      for adoption)
      \item	(organization)’s constitution must have the approval of Student
      Organization Recognition Committee.
      \item	Final approval of (organization)’s constitution must come from the
      Vice Chancellor of Student Affairs for this constitution to be valid and
      take full effect.
    \end{enumerate}
  \item	Amendments:
    \begin{enumerate}
      \item	(Specify process for introducing amendments, i.e. should they be
      submitted in writing, submitted at a meeting or to an officer, etc.).
      \item	(Specify the process and length of notification of members for
      voting on amendments)
      \item	(Specify vote needed to approve amendments)
      \item	Approval of Student Organization Recognition Committee or its
      representative
      \item	Student Involvement will be notified of any amendments voted in, and
      provided with a copy of the changes made.
    \end{enumerate}
  \item	By-laws (If by-laws are called for in the constitution, they should
contain information subject to frequent change.  The constitution should contain
the information not required for frequent revision): Note:  The process for
by-laws may be similar to the process for amendments.  You will want to include
similar information regarding the by-law process as you did for the amendments.
    \begin{enumerate}
      \item	Clause reflecting a need for by-laws.
      \item	Vote needed for adoption
      \item	Vote needed for amending by-laws
      \item	When by-laws may be adopted or revised (special meeting, regular
      meeting, etc)
      \item	Must be provided to the Department of Student Involvement upon
      approval
    \end{enumerate}
\end{enumerate}

