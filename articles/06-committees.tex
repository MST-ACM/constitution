
\section{VI \textendash{} Committees}
Committees are groups that aim to help ACM achieve and expand its overall
mission by adhering to specific goals and interests. These committees operate
independently from ACM as long as their actions and interest do not conflict or
compete with ACM or other committees. Committees must have a chair, but may
internally have their own procedures and governance structure. If there is ever
a point in which a committee does not have a chair, the VP-I will temporarily
take over as chair, assuming all duties and responsibilities, and institute a
special election for a new chair following the election procedures of the
committee. After one month of the absence occurring if no chair is elected, the
Executive Board will vote to dissolve the committee by making a constitutional
amendment to remove the committee from the list of standing committees.

The active standing committees are as follows:
\begin{enumerate}[nolistsep]
  \item ACM Competition (Comp)
    \begin{enumerate}
      \item To learn and encourage competitive programming skills through
      competitions and teaching seminars.
    \end{enumerate}
  \item ACM Data (Data)
    \begin{enumerate}
      \item To develop students ability to explore data through fellowship,
        mentorship, and blogging.
    \end{enumerate}
  \item ACM Game (Game)
    \begin{enumerate}
      \item To help students explore software development and artificial
        intelligence through development and competition.
    \end{enumerate}
  \item ACM Hack (Hack)
    \begin{enumerate}
      \item To encourage students to learn, build, and share through
        hackathons.
    \end{enumerate}
  \item ACM Security (Security)
    \begin{enumerate}
      \item To teach students physical and cyber security skills and best
        practices.
    \end{enumerate}
  \item ACM Women (ACM-W)
    \begin{enumerate}
      \item To build a community of women in computing to help promote
        outreach, diversity, and mentorship.
    \end{enumerate}
  \item ACM Web (Web)
    \begin{enumerate}
      \item To teach students web development through hands on website
        implementation and training.
    \end{enumerate}
\end{enumerate}

\subsection{Creating a Standing Committee}
To create a standing committee (including converting a temporary committee to
standing committee), the person who wishes to create the committee ('prospective
chair') must submit a document to the Executive Board containing the following
elements:
\begin{enumerate}[label=\arabic*., nolistsep]
  \item \textbf{Purpose} - What is the purpose/goal that the committee hopes to
    achieve? What are the benefits that this new committee would bring for ACM
    and for students?
  \item \textbf{Audience} - Who is the audience they are targeting with their
    purpose/goal? How do they initially plan to recruit and on-board these new
    members?
  \item \textbf{Organizational Structure} - How is the committee structured
    through officer positions, elections, meetings, etc.?
  \item \textbf{Initial Members} - A list of people who have expressed interest
    in joining the committee.
\end{enumerate}

There is no length or word requirement as long as the prospective chair
adequately answers these questions. The document will then be given to any S\&T
ACM officer who will communicate it to the entire Executive Board. The
prospective chair will then be given a chance at the next executive meeting to
present this document. If the prospective chair cannot make executive meetings
due to a irreconcilable conflict, a special meeting will be organized so that it
can be presented. After the document is presented, the Executive Board will
privately discuss the new committee and vote by creating a constitutional
amendment to add the new committee.

\subsection{Temporary Committees}
\begin{enumerate}
  \item Temporary committees serve to accomplish long-term ACM projects or
    needs that may not fall under the purview or interest of a specific
    committee. The temporary committee follows the same operating procedures
    as committees mentioned above; however, temporary committee chairs do
    not share the voting rights of committee chairs and are not included in
    Executive Board majority.
  \item Temporary committees can be formed at the discretion of the
    President, Vice President, or a simple majority vote of the Executive
    Board.
  \item Temporary committees will last for a maximum of one year. If the
    temporary committee is necessary after one year, the Executive Board must
    follow the procedure above to convert it to a standing committee.
\end{enumerate}
