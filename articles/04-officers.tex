\section{IV \textendash{} Officers}
\subsection{Officer Positions}
\begin{enumerate}[label=\arabic*.]
  \item President
  \item Vice-President Internal
  \item Vice-President External
  \item Treasurer
\end{enumerate}
\subsection{Executive Board}
\begin{enumerate}[label=\arabic*.]
  \item The Executive Board shall consist of the current S\&T ACM officers and
    all committee chairpersons. A new Executive Board shall take effect
    immediately following the installation of new officers or chairpersons.
\end{enumerate}

\subsection{Qualifications of the officers}
To be eligible for nomination, you must be able to meet these qualifications at or before
time of officer installation:
\begin{enumerate}[label=\arabic*.]
  \item Qualifications for President position
    \begin{enumerate}[label=\arabic*.]
      \item $\geq{2.5}$ cumulative GPA
      \item ACM Member as specified in
        Section~\ref{sec:membership:classes:member}
      \item $\geq{1}$ full semester in active participation on the Executive
        Board of S\&T ACM
    \end{enumerate}
  \item Qualifications for Vice President roles
    \begin{enumerate}[label=\arabic*.]
      \item $\geq{2.5}$ cumulative GPA
      \item ACM member as specified in
        Section~\ref{sec:membership:classes:member}
      % Be sure to update section stuff, CROSS-REFERENCES
      \item $\geq{2}$ full semesters in active participation on one of the
        committees\\ or $\geq{1}$ full semester in active participation
        experience serving in an officer position in any Missouri S\&T RSO
    \end{enumerate}
  \item Qualifications for Treasurer position
    \begin{enumerate}[label=\arabic*.]
      \item $\geq{2.5}$ cumulative GPA
      \item ACM Member as specified in
        Section~\ref{sec:membership:classes:member}
    \end{enumerate}
\end{enumerate}

\subsection{Define the term of the office}
\begin{enumerate}[label=\arabic*.]
  \item All officer positions last until the end of the academic year they are
    elected.
  \item Any one member may not hold the same office for more than two academic
    years, consecutive or non-consecutive.
\end{enumerate}

\subsection{Duties and Responsibilities of Officers (list the duties and
responsibilities of each officer below)}
\begin{enumerate}[label=\arabic*.]
  \item All officers are responsible for ensuring that ACM is acting in
    accordances to its goals and mission.
  \item President:
    \begin{enumerate}[label=\alph*.]
      \item \textbf{Guidance} - The President should be a primary driver of
        innovation through new ideas and processes within ACM to bring more
        value to students and further reinforce S\&T ACM's mission.
      \item \textbf{Official Review} - The President may review any
        communication, official or unofficial, or any decisions made on behalf
        of S\&T ACM or its committees including but not limited to matters of
        sponsorships or branding. Any decision in which official review is
        exercised may be overwritten by a majority vote of Executive Board
        members. This vote can be taken electronically or in person.
      \item \textbf{Task Delegation} - The President may assign necessary
        ACM-related tasks to any officer if there are no volunteers.
      \item \textbf{Student Council Representative} - The President is
        responsible for attending Student Council meetings and representing S\&T
        ACM on campus. Can be delegated to a student council representative.
    \end{enumerate}
  \item Vice President External (VP-E)
    \begin{enumerate}[label=\alph*.]
      \item \textbf{Sponsorship / Corporate Relations} - The VP-E is S\&T ACM's
        primary contact for maintaining corporate relationships and
        establishing new sponsorships with the assistance of the President.
      \item \textbf{Events} - The VP-E is responsible for scheduling,
        organizing, and planning all ACM related events.
    \end{enumerate}
  \item Vice President Internal (VP-I)
    \begin{enumerate}[label=\alph*.]
      \item \textbf{Committee Advocate} - The VP-I acts as ACM's primary
        liaison to the different committees. The VP-I ensures that the
        committees are meeting their deadlines and operating efficiently
        through means such as budget and documentation review. The VP-I
        should work with the President to establish new programs to further
        develop the committees.
      \item \textbf{Compliance} - The VP-I ensures that the Executive Board and
        ACM members are compliant with the constitution and any other by-laws
        governing the actions of the organization.
      \item \textbf{Recording} - The VP-I is in charge of taking notes during
        Executive Board meetings and recording other important ACM moments, and
        if not able to do so, delegating that responsibility.
      \item \textbf{Election Manager} - The VP-I is in charge of all election
        announcements, proceedings, scheduling, and the counting of votes.
    \end{enumerate}

  \item Treasurer
    \begin{enumerate}[label=\alph*.]
      \item \textbf{Budget} - The Treasurer drafts the budget for all of ACM
        and ensures that all committees are compliant with the budget. The
        Treasurer must then communicate this budget to SAFB or any other
        school funding board to acquire school funding. The Treasurer must
        constantly update this budget to account for actual expenditures.
      \item \textbf{Banking} - The Treasurer is in charge of all ACM-based
        banking applications such as the Phelps County bank, Stripe, and
        Square accounts.
    \end{enumerate}
\end{enumerate}
\subsection{Define election process for officers}
\begin{enumerate}[label=\arabic*.]
  \item Elections will occur once per academic year during the spring semester.
    Nominations will begin the last week of March.
  \item Nominations will be made through an online form available to all ACM
    members and committee members. The nomination period shall last one week
    from when the public announcement is communicated.
  \item Once nominations close, there will be an announcement with the list of
    nominees and the time, date, and location of the official elections to all
    voting members.
  \item Voting will occur in person and will be done with secret ballot. At the
    election, each nominee will be given 5 minutes to speak on their experience
    and answer questions from the body. All ACM Members will be able to vote except
    the President of the Executive Board. The winner will be calculated with
    approval voting.
  \item If there is a tie, the current Executive board will leave the room and
    vote in private by simple majority which candidate will be elected. The
    Executive Board may also vote for no confidence, signaling a vacancy in the
    position.
  \item In the case of a sole candidate, the ballot will list two options: the
    candidate and no confidence which will indicate a vote against the
    candidate. Election will be a success if there are more approval than no
    confidence votes. Otherwise, the position will be considered a vacancy.
  \item If a vacancy should arise, the Executive Board will be responsible for
    the election of an interim. Members of the Executive Board will nominate
    willing people for the position and vote for each position by simply
    majority vote of Executive Board members. After an interim is in place for
    two weeks, an online form will be sent out to the ACM Members to confirm or
    deny the interim. The Executive Board may also choose by a 2/3s vote to host
    an official election following the process above to fill the vacancy.
\end{enumerate}

\subsection{Method of impeachment}\label{sec:officers:impeachment}
\begin{enumerate}[label=\arabic*.]
  \item An officer may be impeached for violating S\&T ACM's Rules and
    Regulations outline in Section \ref{sec:rules_and_regulations}.
  \item The impeachment process starts when the person who wishes to impeach
    an officer (`accuser') speaks with an advisor of ACM or an advisor of
    any standing committee of ACM. The accuser must draft a formal letter of
    grievances with evidence or examples of specific violations. The accuser will
    present this letter of grievances to the advisor in an official meeting
    and discuss with the advisor the specific violations. The advisor will
    then determine if there is a valid complaint according to the Rules and
    Regulations. If the complaint is valid, the advisor will set up a
    meeting between all advisors in ACM (including standing committees) to
    discuss the impeachment. The accuser and the accused will be invited to
    this meeting one week before the meeting takes place. The accused will
    received a copy of the formal complaint letter at this time. The accuser and
    the accused will be given 5 minutes to speak and make their case.
  \item The advisors will then privately discuss the case and vote to impeach
    the officer. Three or more advisors must be present and the vote must be
    unanimous. In the case that three or more advisors are not available, the
    department chair will be asked to make the decision with any available
    advisors. If the department chair is unavailable or unwilling, then the
    compliant will be brought up to Missouri S\&T Student Life.
\end{enumerate}
