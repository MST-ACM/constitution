% TODO edit this
\section{IV \textendash{} Officers}
  \subsection{ Officer Positions}
    \begin{enumerate}[label=\arabic*.]
      \item President
      \item Vice-President Internal
      \item Vice-President External
      \item Treasurer
    \end{enumerate}
  \subsection{Executive Board}
    \begin{enumerate}[label=\arabic*.]
      \item The Executive Board shall consist of the current officers, committee
      chairpersons, appointed representatives of committees, and Advisors. The
      new Executive Board shall take effect immediately following the
      installation of officers.
    \end{enumerate}
  \subsection{Qualifications of the officers}
    \begin{enumerate}[label=\arabic*.]
      \item Qualifications for all officer positions
        \begin{enumerate}[label=\arabic*.]
          \item 2.75 minimum cumulative GPA
          \item ACM Member as specified in section III 2.1.a.1
          % Be sure to update section stuff, CROSSREFERENCES
          \item $\geq{1}$ full semester in active participation of one of the
          committees
        \end{enumerate}
      \item Qualifications for President position
        \begin{enumerate}[label=\arabic*.]
          % Be sure to update section stuff, CROSSREFERENCES
          \item $\geq{1}$ full semester in active participation of the executive
          board of ACM
        \end{enumerate}
      \item Qualifications for Vice President roles
        \begin{enumerate}[label=\arabic*.]
          % Be sure to update section stuff, CROSSREFERENCES
          \item $\geq{2}$ full semesters in active participation of one of the
          committees
          \item $\geq{1}$ full semester in active participation experience
          serving in an officer position in any Missouri S\&T RSO
        \end{enumerate}
    \end{enumerate}
 
  \subsection{Define the term of the office}
    \begin{enumerate}[label=\arabic*.]
      \item All officer positions last for one academic year
      \item Any one member may not hold the same office for more than two
      academic years, consecutive or non-consecutive.
    \end{enumerate}
  \subsection{Duties and Responsibilities of Officers (list the duties and
  responsibilities of each officer below)}
    \begin{enumerate}[label=\arabic*.]
      \item All officers are responsible for (insert responsibilities)
      \item President:
        \begin{enumerate}[label=\alph*.]
          \item \textbf{Guidance} - The President should be a primary driver of
            innovation through new ideas and processes within ACM to bring more
            value to students and further reinforce ACM's mission.
          \item \textbf{Official Review} - The President may review any
            communication, official or unofficial, or decision made on behalf of
            ACM or that affects more than committee such as sponsorship or
            branding matters.
          \item \textbf{Task Delegation} - The President may assign necessary
            ACM-related tasks to any officer if there are no volunteers.
        \end{enumerate}
      \item Vice President External (VP-E)
        \begin{enumerate}[label=\alph*.]
          \item \textbf{Sponsorship / Corporate Relations} - The VP-E is ACM's
            primary contact for maintaining corporate relationships and
            establishing new sponsorships with the assistance of the President. 
          \item \textbf{External Events} - The VP-E is responsible for
            scheduling, organizing, and planning all corporate, sponsor, and
            inter-organization events.
        \end{enumerate}
      \item Vice President Internal (VP-I)
        \begin{enumerate}[label=\alph*.]
          \item \textbf{Committee Advocate} - The VP-I acts as ACM's primary
            liaison to the different committees. The VP-I ensures that the
            committees are meeting their deadlines and operating efficiently
            through means such as budget and documentation review. The VP-I
            should work with the President to establish new programs to further
            develop the committees.
          \item \textbf{Compliance} - The VP-I ensure that the executive board
            and ACM members are compliant with the constitution and any other
            bylaws governing the actions of the organization. Along with that,
            the VP-I is in charge of taking notes during Executive Board
            meetings, and if not able to do so, delegating that responsibility.
          \item \textbf{Election Manager} - The VP-I is in charge of all
          election announcements, proceedings, scheduling, and the counting of votes.
          \item \textbf{Internal Events} - The VP-I is responsible for
            organization and planning all non-corporate ACM and intra-committee
            events.
            % TODO revisit this wording
        \end{enumerate}

      \item Treasurer
        \begin{enumerate}[label=\alph*.]
          \item \textbf{Budget} - The Treasurer drafts the budget for all of ACM
            and ensures that all committees are compliant with the budget. The
            Treasurer must then communicate this budget to SAFB or any other
            school funding board to acquire school funding. The Treasurer must
            constantly update this budget to account for actual expenditures.
          \item \textbf{Banking} - The Treasurer is in charge of all ACM-based
            banking applications such as the Phelps County bank, Stripe, and
            Square accounts.
        \end{enumerate}
    \end{enumerate}
  \subsection{Define election process for officers}
    \begin{enumerate}[label=\arabic*.]
      \item Elections will occur once in an academic year during the spring
      semester. Nominations will begin in the first week of April.
      \item Nominations will be made through an online form available to all ACM
      members and committee members. The nomination period shall last one week
      from when the public announcement is emailed to all persons on the ACM
      general email list.
      \item Election announcements will happen the day that nominations go live,
      and again three days following that first announcement. Once nominations
      close, there will be an announcement with the list of nominees. A time and
      date will be included in that announcement with details for the meeting
      which will include the election. All announcements will occur as emails to
      the ACM mailing list, as well as to all of the committee group chats.
      \item Voting will occur in person and will be done with secret ballot. At
      the election, each nominee will be called up to speak on their experience
      and answer questions from the body. ACM Members will be able to vote
      except the president of the Executive Board. The winner will be calculated
      with approval voting. Each ballot will list all names of candidates and a
      checkbox next to each name, if the voter would approve of the candidate
      being elected, they would place a checkmark in the box. At the end, the
      candidate with the largest number of checkmarks will be declared the
      winner.
      \item If there is a tie, the current Executive board will leave the room
      and decide in private and vote by simple majority which candidate will be
      elected. The Executive Board may also vote for no confidence, signalling a
      vacancy in the position.
      \item In the case of a sole candidate, the ballot will list two options,
      the candidate, and no confidence, which will indicate a vote against the
      candidate. Election will be a success if there are more approval than no
      confidence votes. Otherwise, the position will be considered a vacancy.
      \item If a vacancy should arise, either by an elected officer leaving, or
      by no consensus in a vote, the following procedure must be followed. The
      Executive Board will be responsible for the election of an interim. This
      will happen by members of the Board nominating persons for the position,
      and a vote taking place that requires a simple majority of the Executive
      Board to vote in favor. Members of the Executive Board will be given the
      option of voting for, or no confidence. After an interim is in place, an
      online form will be sent out to the ACM Members, and if the members
      confirm the interim, they will become officially part of the board.
    \end{enumerate}
  \subsection{Method of impeachment (Must ensure due process)}
    \begin{enumerate}[label=\arabic*.]
      \item Grounds for removal
      \item Process for removal
      \item Due process for accused (ie the opportunity to speak in their
      defense) and appeals process
      \item Vote \% required to remove the accused.
    \end{enumerate}