\documentclass[11pt,a4paper,notitlepage]{article}
\usepackage[utf8]{inputenc}
\usepackage{amsmath}
\usepackage{amsfonts}
\usepackage{amssymb}
\usepackage{enumitem}
\author{Clay McGinnis}
\title{Constitution of the Missouri University of Science and Technology Student Chapter of the Association for Computing Machinery}
\begin{document}
\maketitle



% TODO: INSERT INSIGNIA
\section*{ARTICLE I - NAME, INSIGNIA, AND AFFILIATION}
\begin{enumerate}[label=\Alph*.]
  \item The name of this organization shall be the Missouri University of
  Science and Technology Student Chapter of the Association for Computing
  Machinery, hereinafter referred to as MST ACM.
  \item The insignia of the organization shall be a blue square with rounded
    corners, tilted 45 degrees with an inner circle, also blue, and a white
    stroke surrounding the inner circle. TODO
  \item MST ACM will be affiliated with the Association for Computing Machinery.
\end{enumerate}


% TODO edit this
\section*{ARTICLE II - AIMS AND OBJECTIVE OF THE ORGANIZATION}
\subsection{Mission}
Enrich the college experience and foster the next generation of innovators by
hosting talks, competitions, and workshops that provide real-world learning
opportunities and networking.
\subsection{Aims and Objectives}
    \begin{enumerate}[label=\arabic*.]
      \item To promote a greater interest in computing machinery and its
      applications.
      \item To provide students with the opportunity to improve their skills in
      computing through real-world experiences.
      \item To create a fair and inclusive community of computing students with
      like-minded interests.
    \end{enumerate}


% TODO edit this
\section*{ARTICLE III - MEMBERSHIP}
\begin{enumerate}[label=\arabic*.]
  \item Qualifications
    \begin{enumerate}[label=\alph*.]
      \item In order to be a member, individuals must meet one of the following
      criteria.
        \begin{enumerate}[label=\arabic*.]
          \item Student at Missouri University of Science and Technology.
          \item Faculty and staff members of Missouri University of Science and
          Technology.
        \end{enumerate}
    \end{enumerate}
  \item Classes of Membership
    \begin{enumerate}[label=\arabic*.]
      \item The following are the classes of membership within MST ACM.
        \begin{enumerate}[label=\alph*.]
          \item Member
            \begin{enumerate}[label=\arabic*.]
              \item All persons who fit the above criteria and have paid their
              local dues are given the right to vote, speak, and run for officer
              positions.
            \end{enumerate}
          \item Advisors
            \begin{enumerate}[label=\alph*.]
              \item All faculty approved by the Board (TODO: add links) to
              advise operations of committees or ACM as a whole. Members of this
              category have the right to speak.
            \end{enumerate}
        \end{enumerate}
    \end{enumerate}
\end{enumerate}

% TODO: Add procedures section for removal of members


% TODO edit this
\section*{ARTICLE IV - Officers}
\begin{enumerate}[label=\Alph*.]
  \item Officer Positions
    \begin{enumerate}[label=\arabic*.]
      \item President
      \item Vice-President Internal
      \item Vice-President External
      \item Treasurer
      \item Secretary/Event Coordinator
      % Rename Secretary
    \end{enumerate}
  \item Executive Board
    \begin{enumerate}[label=\arabic*.]
      \item The Executive Board shall consist of the current officers, committee
      chairpersons, appointed representatives of committees, and Advisors. The
      new Executive Board shall take effect immediately following the
      installation of officers.
    \end{enumerate}
  \item Qualifications of all officer positions
    \begin{enumerate}[label=\arabic*.]
      \item 2.75 minimum cumulative GPA
      \item ACM Member as specified in section III 2.1.a.1
      % Be sure to update section stuff, CROSSREFERENCES
      \item $\geq{1}$ semester in active participation of one of the committees
    \end{enumerate}
  \item Qualifications of President position
    \begin{enumerate}[label=\arabic*.]
      % Be sure to update section stuff, CROSSREFERENCES
      \item $\geq{1}$ semester in the executive board of ACM
    \end{enumerate}
  \item Qualifications of Vice Predident roles
    \begin{enumerate}[label=\arabic*.]
      % Be sure to update section stuff, CROSSREFERENCES
      \item $\geq{2}$ semesters of active participation of one of the committees
      \item $\geq{1}$ semester of experience serving in an officer position in
      any Missouri S\&T RSO
    \end{enumerate}
 
  \item Define the term of the office
    \begin{enumerate}[label=\arabic*.]
      \item All officer positions last for one academic year
      \item Any one member may not hold the same office for more than two
      academic years, consecutive or non-consecutive.
    \end{enumerate}
  \item Duties and Responsibilities of Officers (list the duties and
  responsibilities of each officer below)
    \begin{enumerate}[label=\arabic*.]
      \item All officers are responsible for (insert responsibilities)
      \item President:
        \begin{enumerate}[label=\alph*.]
          \item (insert responsibility)
          \item (insert responsibility)
          \item Etc.
        \end{enumerate}
      \item Vice President
        \begin{enumerate}[label=\alph*.]
          \item (insert responsibility)
          \item (insert responsibility)
          \item Etc.
        \end{enumerate}
      \item Etc.
        \begin{enumerate}[label=\alph*.]
          \item (insert responsibility)
          \item (insert responsibility)
          \item Etc.
        \end{enumerate}
    \end{enumerate}
  \item Define election process for officers
    \begin{enumerate}[label=\arabic*.]
      \item Elections will occur once in an academic year during the spring
      semester. Nominations will begin in the first week of April.
      \item Nominations will be made through an online form available to all ACM
      members and committee members. The nomination period shall last one week
      from when the public announcement is emailed to all persons on the ACM
      general email list.
      \item Election announcements will happen the day that nominations go live,
      and again three days following that first announcement. Once nominations
      close, there will be an announcement with the list of nominees. A time and
      date will be included in that announcement with details for the meeting
      which will include the election. All announcements will occur as emails to
      the ACM mailing list, as well as to all of the committee group chats.
      \item Voting will occur in person and will be done with secret ballot. At
      the election, each nominee will be called up to speak on their experience
      and answer questions from the body. ACM Members will be able to vote
      except the president of the Executive Board. The winner will be calculated
      with simple majority. If there is a tie, all candidates not involved in
      the tie will be removed from the ballot and votes will be cast again, if
      there are only two candidates, the president will be allowed to vote and
      decide. In the case of a sole candidate, the ballot will list two options,
      the candidate, and no confidence, which will indicate a vote against the candidate.
    \end{enumerate}
  \item If a vacancy should arise, either by an elected officer leaving, or by
  no consensus in a vote, the following procedure must be followed. The
  Executive Board will be responsible for the election of an interim. This will
  happen by members of the Board nominating persons for the position, and a vote
  taking place that requires a simple majority of the Executive Board to vote in
  favor. Members of the Executive Board will be given the option of voting for,
  or no confidence. After an interim is in place, an online form will be sent
  out to the ACM Members, and if the members confirm the interim, they will
  become officially part of the board.
  \item Method of impeachment (Must ensure due process)
    \begin{enumerate}[label=\arabic*.]
      \item Grounds for removal
      \item Process for removal
      \item Due process for accused (ie the opportunity to speak in their
      defense) and appeals process
      \item Vote \% required to remove the accused.
    \end{enumerate}
  \end{enumerate}

 % TODO edit this
\section*{ARTICLE V - Advisors}
\begin{enumerate}[label=\Alph*.]
  \item	(Specify how the advisor is selected, i.e. appointment, election,
  selection, etc.
  \item (Specify the how long someone must or can hold this position).
  \item Method of impeachment (Must ensure due process)
    \begin{enumerate}[label=\arabic*.]
      \item Grounds for removal
      \item Process for removal
      \item Due process for accused (ie the opportunity to speak in their
      defense) and appeals process
      \item Vote \% required to remove the accused.
    \end{enumerate}
\end{enumerate}


\section*{ARTICLE VI - Committees}
\begin{enumerate}[label=\Alph*.]
  \item Standing Committees Note:  In this section each Standing Committee
  should be listed, along with who chairs them or how and by whom they are
  appointed, and their purpose.
    \begin{enumerate}
      \item Committee
        \begin{enumerate}
          \item Chair
          \item Purpose
        \end{enumerate}
      \item Committee 
        \begin{enumerate}
          \item Chair
          \item Purpose
        \end{enumerate}
      \item Etc.
    \end{enumerate}
  \item Temporary committees Note:  In this section it should specify how
  temporary committees are appointed, by whom, and their length of existence.
    \begin{enumerate}
      \item How appointed
      \item Length of existence
    \end{enumerate}
\end{enumerate}


\section*{VII - Dues}
Note:  In this section specify whether or not your organization has dues.  If
your organization has dues, in this section it should be specified how the
amount of dues is determined (i.e. vote, national organization, etc.).  It
should also specify when dues are paid, whom they are paid to, and any
consequences that may result from dues not being paid.  Should failure to pay
dues result in a certain membership status (i.e. inactive or non-voting), the
membership category will need to be listed in the membership section of the
constitution.
\begin{enumerate}
  \item	(Specify how dues are determined)
  \item	(Specify when dues must be paid and to whom)
  \item	(Specify any consequences for failure to pay dues)
\end{enumerate}



\section*{VIII - Meetings}
Note:  This section should list all meeting types that your organization has
(i.e. business, regular, special, executive, etc.); who is required or allowed
to attend each meeting type; the frequency of the meetings (i.e. weekly,
monthly, etc.); and, who may call each type of meeting. 
\begin{enumerate}
  \item	Types of meetings
    \begin{enumerate}
      \item	Regular:  (insert description/purpose of meeting, who may call this
      meeting; frequency of meeting type; who is expected or allowed to attend.
      \item	Special
      \item	Etc.
    \end{enumerate}
  \item	Definition of a quorum (specify when quorum is to be utilized, what
  percentage of membership is needed for quorum, and what happens if quorum does
  not exist)
  \item	Parliamentary authority to be used (e.g. Robert’s Rules of Order, Newly
  Revised).  (Specify what parliamentary authority, if any, will be utilized by
  your organization).
\end{enumerate}




\section*{IX - Rules and regulations}
Note:  This section should express that the organization will follow all federal
and state laws and abide by university policies.  If this organization is
affiliated with a national organization, they will also want to express that
they will abide by the policies of the national organization as well.  Lastly,
this section should include a statement on the organizations policy on hazing
and alcohol use.
\begin{enumerate}
\item (Organization) will follow federal and state laws and abide by the rules
of the university
\item (Organization) will follow the rules/regulations of national organization
(if applicable)
\item The Code of Conduct for (Organization) members is the University of
Missouri Code of Conduct and this constitution.
\item This organization is responsible for behaving in a manner compatible with
the university’s standard for student organizations and Title IX federal laws.
All forms of hazing, condoning and sanctioning of physical abuse, sexual
harassment and sexual violence towards prospective or current members are
illegal and will be immediately reported to the Office of Affirmative Action,
Diversity and Inclusion or to the Office of Community Standards and Student
Conduct for review and possible disciplinary action.
\item Alcohol Use – This organization is responsible for abiding by state law
and university policy for use of alcoholic beverages and controlled substances;
unless consumption is prohibited by the organization, guidelines for responsible
use will be set accordingly. {Guidelines may be listed here or in bylaws}. In
accordance with university policy, student organization funds administered
through a university account may not be used to purchase alcohol, alcohol will
not be publicized in promotional materials, and alcohol will not be consumed on
university property without an approved alcohol permit. Failure to comply with
the university’s alcohol policy will be immediately reported to the Office of
Community Standards and Student Conduct for review and possible disciplinary
action.
  \begin{enumerate}
    \item Note: The responsible and legal use of alcohol is not prohibited by
    the University. You may not use University funds to purchase alcohol, you
    may not have alcohol at sponsored events on University property without
    express permission by the Chancellor, you may not publicize alcohol at an
    event in your advertisements. All violations of the University alcohol
    policy and/or illegal use of alcohol MUST be reported to the Office of
    Community Standards. 
  \end{enumerate}
\end{enumerate}

\section*{X. - Constitution}
Note:  This section should specify how the constitution is approved and amended,
and the specific process for introducing, voting on, and approving amendments
and by-laws.
\begin{enumerate}
  \item Adoption
    \begin{enumerate}
      \item	(Specify vote and/or process carried out by the organization members
      for adoption)
      \item	(organization)’s constitution must have the approval of Student
      Organization Recognition Committee.
      \item	Final approval of (organization)’s constitution must come from the
      Vice Chancellor of Student Affairs for this constitution to be valid and
      take full effect.
    \end{enumerate}
  \item	Amendments:
    \begin{enumerate}
      \item	(Specify process for introducing amendments, i.e. should they be
      submitted in writing, submitted at a meeting or to an officer, etc.).
      \item	(Specify the process and length of notification of members for
      voting on amendments)
      \item	(Specify vote needed to approve amendments)
      \item	Approval of Student Organization Recognition Committee or its
      representative
      \item	Student Involvement will be notified of any amendments voted in, and
      provided with a copy of the changes made.
    \end{enumerate}
  \item	By-laws (If by-laws are called for in the constitution, they should
contain information subject to frequent change.  The constitution should contain
the information not required for frequent revision): Note:  The process for
by-laws may be similar to the process for amendments.  You will want to include
similar information regarding the by-law process as you did for the amendments.
    \begin{enumerate}
      \item	Clause reflecting a need for by-laws.
      \item	Vote needed for adoption
      \item	Vote needed for amending by-laws
      \item	When by-laws may be adopted or revised (special meeting, regular
      meeting, etc)
      \item	Must be provided to the Department of Student Involvement upon
      approval
    \end{enumerate}
\end{enumerate}









The essential officers may at their discretion create non-essential officer
positions to delegate tasks to as they see fit. Officers must be Members who are
currently enrolled Students. Election of officers shall take place every year.
All officers shall be elected near the end of the spring semester and shall
begin their term in office effective May 1st. Should a position on the Executive
Council become open during the year, nominations shall be held at the first
meeting following the position becoming open and shall remain open for a week.
An election shall be held at the second meeting following the position becoming
open and shall remain open for a week. A person may hold the office of President
or Vice-President only if he/she is a member who is at least a Junior by
credit-hours during the year in which he/she serves. Nominations will be held at
the second to last meeting of the Spring semester by method of general floor
nomination or by electronic submission if done electronically. The list of
nominations shall he published along with the notice for the last meeting of the
Spring semester. Election of officers for the following academic year shall take
place at the last meeting of the Spring semester. The voting shall be by secret
ballot, starting at the position of President and working downward in rank.
Those members running for the current office shall be allowed time for a short
speech, and then asked to leave the room while the voting takes place. If the
voting takes place electronically, each candidate will submit a bio to be place
online with the ballot instead of giving a short speech. The votes will then be
counted if done in person or revealed if done electronically. In the event that
the membership wishes to impeach an officer or advisor, the impeachment
proceeding must be initiated by a unanimous decision from the remaining members
of the executive board, or in the case of the Advisor, all executive board
members. If an officer is being impeached, a written list of charges must then
be brought to the Advisor, who will review the written charges and make a
decision whether to proceed with the impeachment process. The matter will then
be announced to the membership in the formal meeting announcement, and voted on
at the next meeting. The Advisor must be present at the impeachment meeting and
the written charges must be presented to the general membership. The accused
will then be given the opportunity to make a rebuttal, and a vote will then be
taken. The impeachment must be approved by a two-thirds majority of the voting
membership.







The duties of the President shall be: To call and preside at all general,
special, and executive meetings. To appoint all committees and committee
chairpersons of this Chapter. To raise funding for ACM and all of the SIGs. The
duties of the Vice-President shall be: To temporarily assume the duties of the
President in event of the President's absence. If the absence is permanent, the
Vice-President shall assume the duties of the President until a new president
can be elected. To assume those duties of the President that are delegated to
him/her by the President. Scheduling of events and meetings. The duties of the
Secretary shall be: To keep minutes of all Chapter meetings. To prepare the
Annual Chapter Report for approval by the Executive Council. To be responsible
for all Chapter correspondence. To maintain and archive records of Chapter
activities. Make fliers and publicize all ACM events. The duties of the
Treasurer shall be: To collect dues and maintain all financial records and
membership records. To produce a financial or membership statement when
requested and to report on the Chapter's financial standings at each meeting.
The duties of the Systems Administrator shall be: Maintain an up to date
website. Maintain and document all significant ACM equipment. Documentation
includes a list of all passwords. This document must be accessible to all
members of the Executive Council. For equipment that is maintained by a SIG, the
Systems Administrator shall make sure that the respective SIGs are keeping
documentation. Assist SIGs with technical requests. 



The Executive Council shall act on all matters not requiring full membership
participation including financial matters. Minutes of the Executive Council
meetings shall be available for inspection by any member of this Chapter and
shall be filed with the Chapter records. Should an Executive Council member be
temporarily unable to perform his/her assigned duties, those responsibilities
may be assigned to another Executive Council member by a quorum of the Executive
Council. The Advisor shall be chosen by the current Executive Council at the
time of election of officers. If a new advisor is chosen, he/she shall be voted
up by the Executive Council and ACM general Membership. He/she shall be a
Missouri University of Science and Technology faculty or staff member having
global membership in ACM and having a genuine interest in this Chapter.






\end{document}
